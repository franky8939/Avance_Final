\begin{frame}{}
    \begin{center}
        \LARGE Simulaci\'on
    \end{center}
\end{frame}


\begin{frame}{Simulaci\'on}

\begin{itemize}
    \item Para estudiar el modelo Dark-SUSY de forma experimental y caracterizar las propiedad de la se\~nal  
    se requiere generar muestras de simulaci\'on por el m\'etodo de Monte Carlo 
    \item La simulaci\'on requiere generar varias muestras considerando los par\'ametros m\'as importantes del fot\'on oscuro (masa y tiempo de vida) 
    \item Para generar las muestras simuladas se hace uso de paquetes propios del \'area de altas energ\'ias, los cuales act\'uan de forma sequencial y se encargan de diferentes aspectos del proceso de an\'alisis. 
\end{itemize}
    
\end{frame}


\begin{frame}{Paquetes de Simulaci\'on}
\begin{itemize}
    \item \textbf{Madgraph}: Se encarga de procesar la informaci\'on fundamental del modelo como la masa de las particulas madre, diagramas de Feynman, amplitudes de decaimiento, etc. 
    \item \textbf{Pythia}: Se encarga del proceso de hadronizaci\'on, es decir la interacci\'on entre protones, recombinaci\'on de quarks y formaci\'on de nuevas part\'iculas 
    \item \textbf{Delphes}: Se encarga de simular la respuesta del detector al paso de las part\'iculas simuladas, en este paso se consideran eficiencia de detecciones y se extraen variables como la energia, momento y trayectoria reconstruida de las part\'iculas en cuesti\'on
\end{itemize}

\end{frame}

\begin{frame}{Estrategia de Simulaci\'on}

\begin{itemize}
    \item Para generar las muestras se requiere crear un entorno automatizado (python) el cual pueda ser flexible a las diferentes variaciones del modelo
    \item Los archivos resultantes (formato .root) contienen la informaci\'on (teorica+experimental) para realizar el an\'alisis de datos.  
\end{itemize}

\begin{figure}[h]
\centering
\includegraphics[width=0.8\textwidth]{Imag/proyecto_darksusy2.png}
\caption{Diagrama de Flujo del generador.}
\end{figure}

%Incluir diagrama donde se describa como se inicia la simulacion Madgraph->Pythia->Delphes
    
\end{frame}


\begin{frame}{Muestras simuladas}
    \begin{itemize}
        \item Para generar cada muestra se requiere especificar los siguientes par\'ametros: masa del neutralino ($m_{n_{1}}$), masa del dark neutralino ($m_{n_{D}}$), masa del fot\'on oscuro ($m_{\gamma_{D}}$) y tiempo de vida del foton oscuro ($c\tau_{\gamma_{D}}$) que son los par\'ametros del modelo Dark-SUSY. 
        %\item El numero de muestras simuladas se representa en la siguiente tabla
        \item El n\'umero de muestras simuladas es el resultado de todas las combinaciones posibles de los vectores correspondientes a los par\'ametros de generaci\'on, haciendo un total de $\backsim 15 000$ muestras:
    \end{itemize}

\begin{table}
\begin{footnotesize}
\begin{tabular}{|cl|}
\hline
$V_{m_{n_{1}}}$ & = $[10, ~20, ~30, ~40, ~50, ~60, ~70, ~80, ~90, ~100]$\\
$V_{m_{n_{D}}}$ & = $[0.25, ~1, ~2, ~3, ~4, ~5, ~10]$\\
$V_{m_{\gamma_{D}}}$ & = $[0.25, ~1, ~2, ~3, ~4, ~5, ~6, ~7, ~8, ~9, ~10]$\\
$V_{c\tau_{\gamma_{D}}}$ & = $[0, ~1, ~2, ~3, ~4, ~5, ~10, ~20, ~30, ~40, ~50, ~100]$\\
Eventos simulados &  = 10000\\
\hline
\hline
\end{tabular}
\end{footnotesize}
\end{table}
\end{frame}




%\begin{comment}

%\begin{frame}{Muestras simuladas}
%Notaci\'on:\\
%$\mathbb{E}_i^{(j,~k)} ~= ~1$  hace referencia a los eventos, donde $i = \{1, \ldots, i_{max}\}$ corresponde al elemento del evento en el \'arbol de archivo $*.root$ requerido, y donde: $\{j, ~k\} = \{\{0\mu, ~1\mu, ~2\mu, ~3\mu, ~4\mu\, \ldots\},\{\mathtt{CMS},~\mathtt{HL}\}\}$ hace referencia a los eventos seg\'un su contenido mu\'onico y al detector que gener\'o los datos. Entonces:
%\begin{equation*}
%\mathbb{E}^{(j,~\mathtt{CMS})} = \sum_i \mathbb{E}_i^{(j, ~\mathtt{CMS})} , ~~~~ \mathbb{E}_i^{(\mathtt{CMS})} = \sum_j \mathbb{E}_i^{(j,~\mathtt{CMS})} ~~~~ y ~~~~ \mathbb{E}^{\mathtt{(CMS)}}= \sum_{ij} \mathbb{E}_i^{(j,~\mathtt{CMS})}
%\end{equation*}
%\begin{equation*}
%\mathbb{E}^{(j,~\mathtt{HL})} = \sum_i \mathbb{E}_i^{(j, ~\mathtt{HL})} , ~~~~ \mathbb{E}_i^{(\mathtt{HL})} = \sum_j \mathbb{E}_i^{(j,~\mathtt{HL})} ~~~~ y ~~~~ \mathbb{E}^{\mathtt{(HL)}}= \sum_{ij} \mathbb{E}_i^{(j,~\mathtt{HL})}
%\end{equation*}
%donde:
%\begin{equation*}
%\mathbb{E}^{(j,~k)}\equiv \mathbb{E}^{(j,~k)}\mathtt{(MNeuL,~MNeuD,~MPhoD,~TcPhoD)}
%\end{equation*}
%
%\end{frame}



%\begin{frame}{Muestras simuladas}
%Notaci\'on:\\
%\begin{equation*}
%f^{(j,~k)} = \dfrac{\mathbb{E}^{(j,~k)}}{\mathbb{E}^{(k)}}
%\end{equation*}
%Para el caso que nos ocupa $\mathbb{E}^{(k)} ~= ~\mathtt{Event} ~= ~10 000$ son los eventos simulados para cada configuraci\'on requerida. Ejemplos:
%\begin{table}
%\begin{footnotesize}
%\begin{tabular}{|cccccc|}
%\hline
%$\texttt{MNeuL(GeV)}$ & $\texttt{MNeuD(GeV)}$ & $\texttt{MPhoD(GeV)}$ & $\texttt{TcPhoD(mm)}$ & $f^{(4\mu,~\texttt{CMS})}$ & $f^{(4\mu,~\texttt{HL})}$ \\
%\hline
%10 & 0.25 & 0.25 & 0.5 & 0.0920 & 0.1678\\
%& & & 2 & 0.0779 & 0.1355 \\
%& & & 4 & 0.0597 & 0.1024 \\
%& & & 10 & 0.0227 & 0.0433 \\
%& & & 50 & 0.0016 & 0.0039 \\
%\hline
%10 & 0.25 & 0.25 & 2 & 0.0497 & 0.1135 \\
%& & 4 & & 0.0494 & 0.1157 \\
%& & 6 & & 0.0599 & 0.1456 \\
%& & 8 & & 0.0957 & 0.1960 \\
%\hline
%\end{tabular}
%\end{footnotesize}
%\end{table}

%\begin{table}[]
%    \centering
%    \begin{tabular}{c|c|c|}  \hline
%    Masa [GeV]     & Tiempo de vida [mm]  &  Numero de eventos \\ \hline 
%         &  & \\ \hline 
%         & & \\ \hline 
%    \end{tabular}
%    \caption{Muestras simuladas del modelo Dark-SUSY}
%    \label{tab:my_label}
%\end{table}  
%\end{frame}
%\end{comment}


\begin{frame}{Simulaci\'on del Detector en Delphes}

\begin{itemize}
    \item Delphes es un paquete de simulaci\'on rapida, es decir alguna de las eficiencias de deteccion estan parametrizadas, lo anterior para reducir el tiempo de simulaci\'on 
    \item Dichar parametrizaciones se obtienen de la simulaci\'on mas detallada (Geant4) la cual contempla todos los procesos fundamentales del paso de las particulas por el detector
    \item En nuestro modelo los fotones oscuros decaen a muones, por lo que la reconstrucci\'on de los mismos es parte fundamental del an\'alisis
\end{itemize}
\end{frame}


\begin{frame}{Identificaci\'on de muones en CMS}
    
\begin{itemize}
\item Los muones son particulas elementales que interaccionan d\'ebilmente con la materia, su trayectoria se reconstruye con informaci\'on obtenida de detectores dedicados a esta tarea
\item A partir de la trayectoria reconstruida y la desviaci\'on provocada por el campo magn\'etico solenoide se puede obtener el valor del momento.
\item Debido a que el fot\'on oscuro puede viajar una distancia considerable antes de decaer la eficiencia de reconstrucci\'on de muones esta \'intimamente ligado al tiempo de vida (c$\tau$) 
\item En general se espera que a mayor tiempo de vida del fot\'on oscuro la eficiencia de identificaci\'on es menor (debido a que los muones logran atravesar solo una parte de los dispositivos de detecci\'on)
\end{itemize}
\end{frame}


\begin{frame}{Identificaci\'on de muones en CMS}

\begin{itemize}
    \item La linea azul representa el paso de los muones por el detector CMS
\end{itemize}

\begin{figure}
    \centering
    \includegraphics[width=0.8\textwidth]{Imag/reconstruccion_muones.png}
    \caption{Identificaci\'on de part\'iculas en CMS, vista transversal del detector}
    \label{fig:my_label}
\end{figure}
    
\end{frame}



\begin{frame}{Definici\'on de variables relevantes}
El detector CMS tiene una simetr\'ia cilindrica en donde el eje de colisi\'on es el ``z'' y el plano transversal esta formado por los ejes ``x'' e ``y''. Dos de las variables para caracterizar las propiedades de las part\'iculas son las siguientes. 
\begin{figure}[ht]
\begin{minipage}[b]{0.45\linewidth}
\centering
\includegraphics[width=\textwidth]{Imag/pseudorapidity.PNG}
\caption{Definici\'on de pseudorapidez}
\end{minipage}
\hspace{0.5cm}
\begin{minipage}[b]{0.45\linewidth}
\centering
\includegraphics[width=\textwidth]{Imag/pt.PNG}
\caption{Definici\'on de momento transversal}
\end{minipage}
\end{figure} 

\end{frame}



\begin{frame}{Parametrizaci\'on de eficiencia y resoluci\'on}

Estas parametrizaci\'on est\'an contenidas en la simulaci\'on Delphes
\begin{figure}[ht]
\begin{minipage}[b]{0.45\linewidth}
\centering
\includegraphics[width=\textwidth]{Imag/Momentum_resolution_of_Muon_CMS.png}
\caption{Resoluci\'on del momento para muones.}
\end{minipage}
\hspace{0.5cm}
\begin{minipage}[b]{0.45\linewidth}
\centering
\includegraphics[width=\textwidth]{Imag/Eficiencia_of_Muon_CMS.png}
\caption{Eficiencia en la reconstrucci\'on de los muones.}
\end{minipage}
\end{figure}
    
\end{frame}